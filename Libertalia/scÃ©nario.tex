% Créé par Martin Bodin (2014).
% Document sous licence CC BY-NC-SA

% Créé par Martin Bodin (2011).
% Document sous licence CC BY-NC-SA

\documentclass{article}
%\documentclass{scrartcl}

\usepackage{ifxetex}
\ifxetex
\usepackage{xunicode,fontspec,xltxtra}
\else
\usepackage[utf8x]{inputenc}
\usepackage[T1]{fontenc}
\usepackage{amsmath, amsthm}
\usepackage{amsfonts, amssymb}
\fi

\usepackage[francais]{babel}
\usepackage{lmodern}
\usepackage{stmaryrd}
\usepackage{graphicx}
\usepackage[nottoc, notlof, notlot]{tocbibind}
\usepackage[dvipsnames]{pstricks}
\usepackage{pst-circ, pst-plot, pstricks-add}
\usepackage{array}
\usepackage{url}
\usepackage{verse}
\usepackage[colorlinks,linkcolor=black]{hyperref}
\usepackage{ifthen}
\usepackage{longtable, rotating}
%\usepackage{fancyhdr}
\usepackage{fancybox, framed}
\usepackage{textcomp}
\usepackage{marvosym}
%\usepackage{bbding}
%\usepackage{a4wide}
\usepackage{geometry}
%\usepackage{soul}
\usepackage{lettrine}
%\usepackage{yfonts}
\usepackage{oldgerm}
\usepackage{enumerate}
\usepackage{tikz}
\usepackage{dictsym}
\usepackage{pifont}

\ifxetex
\newfontfamily\timesfont[Ligatures=TeX]{Times New Roman}
\setmainfont[Mapping=tex-text, Ligatures={Contextual, Common, Historical, Rare, Discretionary}, Numbers={OldStyle}]{Linux Libertine O}
\fi

%\newcommand{\enluminure}[2]{\lettrine[lines=3]{\small \initfamily #1}{#2}}

\usetikzlibrary{trees}
\usetikzlibrary{arrows,shapes,automata,petri}
\usetikzlibrary{fit}
\usetikzlibrary{calc,decorations.pathmorphing,patterns}


\geometry{
	includeheadfoot,
	margin = 2.54cm,
	top = 1.5cm,
	bottom = 1.5cm
}

\newcommand{\ds}{\displaystyle}

\renewcommand{\ge}{\geqslant}
\renewcommand{\le}{\leqslant}
\renewcommand{\preceq}{\preccurlyeq}
\renewcommand{\succeq}{\succcurlyeq}

\newcommand{\Numero}{\No}
\newcommand{\numero}{\no}

\newcommand{\fixme}{\textbf{FIXME}}

\makeatletter

\newcommand{\defineNewPlayer}[2]{
	\@namedef{couleur#1}{#2}
}

\newcommand{\getPlayerColor}[1]{%
	\@nameuse{couleur#1}%
}

\makeatother

% Des commandes pratiques pour générer le document.
\newcommand{\player}[2]{%
	\ifthenelse{\equal{\forplayer}{y}}{%
		\ifthenelse{\equal{\theplayer}{#1}}%
		{#2}{}%
	}{\begin{barv}[\getPlayerColor{#1}]{2pt}{10pt}#2\end{barv}}%
}
\newcommand{\mj}[1]{%
	\ifthenelse{\equal{\forplayer}{n}}{#1}{}%
}
% Ici suit une commande plus complexe, car plus générale.
\makeatletter

\newcommand{\@beginColor}[3][black]{%
	\ifthenelse{\equal{\forplayer}{n}}{%
		\begin{barv}[#1]{#2}{#3}%
	}{}%
}

\newcommand{\@endColor}{%
	\ifthenelse{\equal{\forplayer}{n}}{%
		\end{barv}%
	}{}%
}


\newcommand{\ignore}[1]{}
\newcommand{\@ident}[3]{%
%	\ifthenelse{\equal{\manyColored}{y}}{#1}{%
%		\marginpar{%
%			#1%
%			\vspace{2cm}%
%			#2%
%		}%
%	}%
	#1%
	\ifthenelse{\equal{\forplayer}{n}}{\@beginColor{0pt}{10pt}}{}%
	#3%
	\ifthenelse{\equal{\forplayer}{n}}{\@endColor}{}%
	#2%
%	\ifthenelse{\equal{\manyColored}{y}}{#2}{%
%		\marginpar{%
%			#1%
%			\vspace{5pt}%
%			#2%
%		}%
%	}%
}

\def\@ouverture#1#2{%
\ifthenelse{\equal{\forplayer}{y}}{}{%
\ifthenelse{\equal{\manyColored}{y}}{\@beginColor[#1]{1pt}{0pt}}{%
\hspace{-1cm}\hspace{-#2mm}\parbox[c][1pt][t]{0pt}{
\begin{tikzpicture}
	\node (a) {};
	\node (b) [right of = a, node distance = 16cm] {};
	\node (c) [below of = a, node distance = 2cm] {};
	\draw [very thick, color = #1] (a.center) -- (b);
	\draw [very thick, color = #1] (a.center) -- (c);
\end{tikzpicture}
}\vspace{-3.2mm}\par%
}%
}%
}
\def\@fermeture#1#2{%
\ifthenelse{\equal{\forplayer}{y}}{}{%
\ifthenelse{\equal{\manyColored}{y}}{\@endColor}{%
\hspace{-1cm}\hspace{-#2mm}\parbox[c][1pt][b]{0pt}{
\begin{tikzpicture}
	\node (a) {};
	\node (b) [right of = a, node distance = 16cm] {};
	\node (c) [above of = a, node distance = 1cm] {};
	\draw [very thick, color = #1] (a.center) -- (b);
	\draw [very thick, color = #1, dashed] (a.center) -- (c);
\end{tikzpicture}
}\vspace{-3.2mm}\par%
}%
}%
}

\def\players@parse#1#2[#3][#4]{%
% #1 :  Suite de \@ouverture
% #2 :  Suite de \@fermeture
% #3 :  Commande à appeler dans le cas d’une réponse négative (≃ réponse précédente).
% #4 :  Argument (sous forme de numéro de joueur) lu actuellement.
	\ifthenelse{\equal{\theplayer}{#4}}{%
		\players@yes{\@ouverture{\getPlayerColor{#4}}{#4}#1}{#2\@fermeture{\getPlayerColor{#4}}{#4}}%
	}{%
		#3{\@ouverture{\getPlayerColor{#4}}{#4}#1}{#2\@fermeture{\getPlayerColor{#4}}{#4}}%
	}%
}

\def\players@no#1#2{%
	\@ifnextchar[{\players@parse{#1}{#2}[\players@no]}{\ignore}%
}

\def\players@yes#1#2{%
	\@ifnextchar[{\players@parse{#1}{#2}[\players@yes]}{\@ident{#1}{#2}}%
}

\def\players{%
	\ifthenelse{\equal{\forplayer}{y}}{%
		\players@no{}{}%
	}{%
		\players@yes{}{}%
	}%
}

% \players{…} est quasi-équivalent à \mj{…}.
% \players[i]{…} est équivalent à \player{i}{…}
% \players[i][j][k]{…} va créer du contenu uniquement pour les joueurs i, j et k (et les MJ bien sûr).

\makeatother
%\fixme :  Ces commandes posent des problèmes pour toutes les sections, footnote, etc. :S

\newcommand{\colorForMJ}[2]{%
	\ifthenelse{\equal{\forplayer}{y}}{#2}{%
		\textcolor{\getPlayerColor{#1}}{#2}%
	}%
}
\newcommand{\synopsisPerso}[3]{%
\paragraph{}{
\textbf{\fcolorbox{\getPlayerColor{#1}}{white}{#2}}\hspace{10pt}%
{#3}}%
}

\newenvironment{changemargin}[2]{\begin{list}{}{%
\setlength{\topsep}{0pt}%
\setlength{\leftmargin}{0pt}%
\setlength{\rightmargin}{0pt}%
\setlength{\listparindent}{\parindent}%
\setlength{\itemindent}{\parindent}%
\setlength{\parsep}{0pt plus 1pt}%
\addtolength{\leftmargin}{#1}%
\addtolength{\rightmargin}{#2}%
}\item }{\end{list}}
\reversemarginpar
%\pagestyle{fancy}
%\fancyhf{}
%\renewcommand{\headrulewidth}{0pt}
%\lhead{}
%\lfoot{}

\makeatletter
\newenvironment{barv}[3][black]{%
% #2 largeur du trait
% #3 distance entre le trait et le texte
	\def\FrameCommand{{\color{#1}\vrule width #2}
	\hspace{#3}}%
	\MakeFramed {\advance \hsize -\width \FrameRestore }%
}{%
    \endMakeFramed%
}
\makeatother


\definecolor{LightButter}{rgb}{0.98,0.91,0.31}
\definecolor{LightOrange}{rgb}{0.98,0.68,0.24}
\definecolor{LightChocolate}{rgb}{0.91,0.72,0.43}
\definecolor{LightChameleon}{rgb}{0.54,0.88,0.20}
\definecolor{LightSkyBlue}{rgb}{0.45,0.62,0.81}
\definecolor{LightPlum}{rgb}{0.68,0.50,0.66}
\definecolor{LightScarletRed}{rgb}{0.93,0.16,0.16}
\definecolor{Butter}{rgb}{0.93,0.86,0.25}
\definecolor{Orange}{rgb}{0.96,0.47,0.00}
\definecolor{Chocolate}{rgb}{0.75,0.49,0.07}
\definecolor{Chameleon}{rgb}{0.45,0.82,0.09}
\definecolor{SkyBlue}{rgb}{0.20,0.39,0.64}
\definecolor{Plum}{rgb}{0.46,0.31,0.48}
\definecolor{ScarletRed}{rgb}{0.80,0.00,0.00}
\definecolor{DarkButter}{rgb}{0.77,0.62,0.00}
\definecolor{DarkOrange}{rgb}{0.80,0.36,0.00}
\definecolor{DarkChocolate}{rgb}{0.56,0.35,0.01}
\definecolor{DarkChameleon}{rgb}{0.30,0.60,0.02}
\definecolor{DarkSkyBlue}{rgb}{0.12,0.29,0.53}
\definecolor{DarkPlum}{rgb}{0.36,0.21,0.40}
\definecolor{DarkScarletRed}{rgb}{0.64,0.00,0.00}
\definecolor{Aluminium1}{rgb}{0.93,0.93,0.92}
\definecolor{Aluminium2}{rgb}{0.82,0.84,0.81}
\definecolor{Aluminium3}{rgb}{0.73,0.74,0.71}
\definecolor{Aluminium4}{rgb}{0.53,0.54,0.52}
\definecolor{Aluminium5}{rgb}{0.33,0.34,0.32}
\definecolor{Aluminium6}{rgb}{0.18,0.20,0.21}

\pgfdeclarelayer{foreground} 
\pgfdeclarelayer{background} 
\pgfsetlayers{background,main,foreground} 



\newcommand{\forplayer}{n} % or y
\newcommand{\theplayer}{10} % if \equal{\forplayer}{y}, then it represents the number of this player.
\newcommand{\manyColored}{n}

\defineNewPlayer{1}{Red} % Rodrigues, le dieu isolé
\defineNewPlayer{2}{Blue} % Mayo Manookian, le protecteur de la Terre
\defineNewPlayer{3}{OliveGreen} % Arman Vezo, le prêtre aux oiseaux
\defineNewPlayer{4}{Cyan} % Lorenz Orazio, le sédentaire
\defineNewPlayer{5}{Brown} % Klaas Goch, l’explorateur
\defineNewPlayer{6}{Yellow} % Angelo Caraccioli, le missionnaire
\defineNewPlayer{7}{Plum} % Olivier Misson, le révolutionnaire
\defineNewPlayer{8}{Gray} % José Felipe, le boulet
\defineNewPlayer{9}{BlueViolet} % Horace Ravin, l’amasseur
\defineNewPlayer{10}{Rhodamine} % Philippe Cosnac, l’infiltré
\defineNewPlayer{11}{Violet} % Étienne Pallière, le docteur


\newcommand{\pageForPlayer}[3]{%
\player{#1}{
	\mj{\newpage}%
	\section{Ton personnage~: #2}
	{#3}
}}

\title{La Fin de l’utopie}
\author{\textsc{Martin Bodin}}
\date{}

\begin{document}

\mj{\maketitle}

%\tableofcontents

%\newpage

\players[6][7][8][9][10]{
	Le capitaine \textsc{Misson}, fils de famille bourgeoise, est cultivé et provençal.  En 1652, il quitte son habit de mousquetaire pour servir le roi de France comme corsaire.  Il embarque donc sur la \textsc{Victoire} comme simple officier de marine, lors d'une escale à Rome, il rencontre \textsc{Caraccioli} et l’enrôle sur la \textsc{Victoire}.

    Faisant route vers les \textsc{Antilles}, la \textsc{Victoire} se fait interceptée par le \textsc{Winchelsea}. La bataille fut difficile, la \textsc{Victoire} perdit son capitaine et bien d'autres, mais une explosion à bord du \textsc{Winchelsea} leur sauve la vie. \textsc{Misson} s'autoproclame capitaine de la \textsc{Victoire} avec l'assentiment de \textsc{Caraccioli}.

    Ensemble, ils décidèrent de partir vers l'Océan Indien, afin de découvrir une terre d'asile pour fonder une société démocratique. Ils finirent par trouver une superbe baie sur une île qu’ils découvrent, \textsc{Rodrigues} (nommée du nom du marin qui l’a découvert) pour enfin  y fonder \textsc{Libertalia}.

	Enfin, ça, c’est ce que l’on racontera plus tard.  En pratique, l’arrivée sur l’île a été plus qu’agitée~:  en naviguant sur l’océan un vent soudain s’est mis à se lever.
	Étonnant de constater une levée si rapide de grands vents en plein océan… surtout ici, si loin de l’\textsc{Afrique}, sur la course du \textsc{Gulf Stream} des \textsc{Indes}, qui apporte ses courants chauds vers le Sud.

	Les deux bâtiments, celui de \textsc{Misson} et celui de \textsc{Tew} se sont retrouvé entraînés contre leurs grés, malgré le combat acharné des marins, droit dans cette direction où… Terre~!  Terre~!  Criait \textsc{Rodrigues}, l’idiot du bateau regardant au loin plutôt que d’aider le reste des marins.  Une île sortait des nuages orageux.
	Malheureusement les deux bateaux se sont retrouvés séparés dans la tempête et chacun ont échoués à plusieurs lieux l’un de l’autre.

	Une fois sur l’île, les hommes ont débarqués sur la baie pour y construire un camp de fortune, en attendant un vent plus clément.
	Malheureusement depuis ce temps là, le vent n’a jamais changé de direction.  Cela va faire des mois et des mois que l’on a pas pu reprendre large ici~!  Bientôt une année…  Ce n’est certes plus une tempête, mais impossible d’utiliser le navire.  Les voyages en barques d’un campement à l’autre sont encore possibles, mais très difficiles, ce qui fait que les contacts entre les deux campements sont très rares.

	Une fois sur l’île, les hommes ont eu deux types de réactions~:
	\begin{itemize}
		\item Certains venaient d’une vie de marin non voulue, enrôlés pour une guerre ou par manque d’argent et ne se rendant pas compte combien la terre est peu vue sur l’océan.  D’autres venaient tout juste d’être libérés de l’esclavage par le capitaine \textsc{Misson} et redécouvrait voire découvrait la signification de la liberté.
			Ces hommes se sont empressés d’utiliser l’échappatoire à la marine qui leur était proposé pour prendre des «~vacances mérité~» dans ce camp de fortune.
		\item D’autres voulaient reprendre le rythme militaire.  Parfois par habitude, mais surtout car les plus cultivés savaient se qui s’était passé leur de la colonisation de l’Amérique~:  les premières années, le cannibalisme n’était pas rare, dû au manque de nourriture, lui même dû au fait qu’aucun champ de céréales n’était prêt les premiers mois de la colonisation.
			Ces hommes ont tout fait pour réorganiser une société hiérarchisé dans le camp… est c’est probablement grâce à eux que celui-ci existe encore.
	\end{itemize}
	Dans tous les cas, aucune exploration sérieuse de l’île n’a pu être exécutée… et tous les hommes envoyés chercher un chemin terrestre entre les deux campements sont soit rentrés bredouille, soit pas rentrés du tout.

	Une société de fortune a donc été mise en place par ces fugitifs, autoproclamés «~hommes libres~».
}

\pageForPlayer{1}{Rodrigues, le dieu isolé}{
        Au commencement était l’Océan.  Une étendue d’eau sans fin, peuplée de créatures chaotiques et mystérieuses.
        De l’Océan naquit l’Île et de l’Île nous naquîmes.
        Nous étions Elle et Elle était Nous.
        Autour était l’Ailleurs.  Il n’y avait rien.  Nous Nous n’y sommes jamais intéressé.

        Nous étions seuls dans l’Océan.  Nous n’avions pas de but spécifique.
        Nous avons donc sculpté l’île.  À partir d’un simple cratère de volcan, Nous avons ajouté des étendues de sable fin, quelques collines, une petite baie.
        À ces collines, Nous avons même planté des arbres.  Nous avons dessiné des fruits.
        Il y avait de l’eau partout.  C’était facile.  À l’Île, Nous lui avons donné un nom :  \textsc{Rodrigues}.
        C’est un beau nom.  Nous ne savons pas ce que cela signifie, mais c’est le nom que Nous avons trouvé, et Nous n’allons pas Nous contester.  Jamais.

        Un jour, quelque chose de nouveau arriva :  quelque chose vint de l’Ailleurs.
        C’était une chose volante, poussant quelques cris de temps en temps, et attrapant des poissons de l’Océan sans se poser.
        Nous étions curieux et Nous l’avons apprivoisé.
        C’était un oiseau, le premier que Nous accueillons et que Nous chérissions.

        Bientôt, d’autres oiseaux vinrent.  Ils étaient beaux.  Et ils volaient.
        Nous modifions l’Île pour subvenir à leurs besoins.
        Une crête par là, pour aider leur envol, une pente douce par ci, pour les protéger du vent, une forêt leur apportant tous les fruits qu’ils chérissaient tant.
        Ils étaient bien ici.  Et Nous adorions les regarder voler toute la journée.
        Parfois Nous jouions avec eux, soufflant une petite brise le long de leur torse, comme une caresse, faisant sortir des poissons de l’Océan, faisant jouer les nuages pour éclairer et chauffer leurs abris.

        Parfois, ils partaient, loin.  C’était leur saison des amours, où ils allaient voyager loin dans l’Océan, loin de Nous.
        Nous ne savons pas où ils vont, mais cela a l’air important.  Peut-être y a-t-il d’autres îles dans l’Océan ?  D’autres Nous ?
        Nous les jalousions, car Ils gardaient Nos oiseaux pendant tout une saison.
        Mais Ils devaient être loin.  L’Océan est infini, tout comme le courage de Nos oiseaux.
        Et puis un beau jour, ils revenaient !  Nous les reconnaissions.
        Les jaunes au plumage de soie et au bec d’argent.  Les verts tachetés de rouge.  Et les bleu iridescents :  c’était eux !
        C’était toujours la grande fête lorsque Nous retrouvions Nos chers amis, partis loin dans l’Océan, mais bravement revenus.

        Un jour, quelque chose d’autre arrivait par l’Océan.
        Ils n’arrivaient pas par les airs, mais par la mer.  Dans un bateau de bois, des humains découvraient Nos terres.
        Ils s’émerveillaient devant Nos oiseaux et sont restés sur Notre île.
        Nous les avons surveillés, curieux.

        Ils étaient débrouillards.
        Très différents des oiseaux, ils coupaient quelques arbres pour concevoir des cabanes très différentes des nids aviaires.
        Mais ils étaient respectueux de la nature.
        Nous sommes ainsi rentrés en contact avec eux.  Et ils Nous ont écoutés.

        Ils comprenaient ce que Nous étions.  Et ils comprenaient qu’il était important de Nous obéir.
        Nous voulions les aider et Nous avons encore remodelé l’île pour qu’elle s’adapte à leurs besoins.
        Mais nous ne voulions pas abîmer les oiseaux, que Nous chérissions toujours autant.
        Les humains n’étaient qu’une de nos curiosités.

        Mais ils avaient l’air ingénieux.  Il Nous est venu à l’esprit que peut-être, ils feraient de bon êtres volants.
        Ils seraient très différents des oiseaux :  au lieu de voler d’eux-mêmes, ils construiraient une machine volante, similaire à leur machine flottante qui les avait amené jusqu’ici.
        Ils commencèrent à étudier Nos oiseaux.  Comment ils volaient, comment ils vivaient.
        Ils faisaient des expériences.  Nous les aidions.
        Nous faisions pousser des arbres aux grandes feuilles larges, ainsi que des fougères, dont les feuilles ressemblent à un plumage fourni.
        Nous nous inspirions de la faune pour créer la flore.  Quelle merveille de la nature Nous avions alors créé !

        Mais les expérimentations étaient longues.
        Et toutes leurs communications étaient orales :  une fois les anciens morts, la plupart était à refaire, tout ce qui n’avait pas été transmis oralement étant perdu.
        Cela dura plusieurs générations.  Nous avons alors trouvé une solution.
        Nous décidions alors de donner aux meilleurs d’entre eux le don d’immortalité.
        Ils entreraient alors dans une phase de création sans fin\ldots jusqu’à ce que la première machine volante s’élève vers le ciel autour de Nous.

        Il y a eu une certaine agitation lorsque Nous avons commencé à répandre la nouvelle, mais Nous les avons rappelé à l’ordre.
        Le processus de sélection durerait probablement des années.
        Nous voulions être certain que seuls ceux qui seraient prêt à travailler sans arrêt à Notre rêve d’élévation soient remerciés.
        La mortalité fait partie des choses de ce monde que l’on ne peut pas enlever sans risque.
        Si Nous ne faisons pas attention, un don trop prompt pourrait compromettre l’île toute entière.
        Nous avons donc pris Notre temps.  Nous n’avons pas encore choisi.  Bientôt, Nous avons des idées, mais il n’est pas encore temps.

        Nous avons alors senti une présence étrangère autour de l’île.
        D’autres humains.  Mais très différents.
        Au lieu d’avoir une embarcation faite de quelques branchages, ces derniers ont des bâtiments complets, très impressionnants.
        Ils sont moins vifs que mes autres humains ou que mes oiseaux, mais ils ont l’air certains d’eux.
        Ils se comportent étrangement, portant sur eux des habits colorés et se comportant très bruyamment par rapport aux autres humains.
        Finalement, peut-être qu’il se comportent plus comme des oiseaux :  à se parader, chacun ayant une couleur bien précise, et leur couleur leur donnant une autorité dans le navire.
        Mais quelque chose en Nous Nous fait sentir qu’ils sont plus\ldots lourds, moins décidés à voler.

        Mais Nous voulions essayer.  Nous avons toujours été curieux.
        Nous avons modifié le vent, et ils sont venu vers l’île.  Pourtant, il semblait qu’ils résistaient.
        Ils ont commencé par faire le tour de l’île, malgré Notre souffle qui les dirigeait droit vers elle !
        Ils ne l’ont probablement pas vu, d’ailleurs, ils étaient encore assez loin.
        Nous étions curieux.  Ils semblaient avoir une volonté propre beaucoup plus forte que tout ce que Nous avions vu jusqu’alors.
        Ils semblaient animés par un autre Nous plus fort\ldots mais plus distant.
        Nous ne sommes donc effectivement pas seul sur l’Océan.

        Nous les haïssions maintenant, avant, ces autres Nous étaient loins.  Voilà qu’Ils viennent à mon contact\ldots et Ils semblent brutaux.
        Il Nous semble important de les convertir.  Si Nous y arrivons, Nous comprendrons les autres Nous, Nous comprendrons comment les combattre.
        Et peut-être comprendrons Nous comment faire voler l’humanité, finalement ?  Ce rêve tant convoité jusqu’alors.

        Nous avons décidé de Nous introduire parmi eux.
        Nous avons créé un des leurs.  Et les autres le connaissaient.
        Nous n’étions pas certains du fonctionnement de Notre pouvoir, mais il semble bien fonctionner.

        Nous Nous sommes vu projeté dans ce nouveau venu.
        Nous avons vu d’autres bateaux, des ports, des villes, des continents !  C’était traumatisant.
        L’Océan était certes infini, Nous ne l’imaginions pas si vaste !

        Les autres Nous surnommaient \textit{l’ahuri}.  Celui qui s’étonne de tout ce qu’on lui dit.
        Il faut dire, que faire d’autre dans ce monde peuplé de\ldots de\ldots de Nous si variés et puissant~!
        Nous étions en haut du mât et Nous avons crié «~Terre~»~!
        Un des autres a alors dit qu’ils appelleraient l’île «~\textsc{Rodrigues}~», car s’était lui qui l’avait vu le premier.
        Lui qui l’avait vu le premier\ldots mais c’est Nous~!  Nous Nous appelons donc \textsc{Rodrigues}, comme Nous — comme \emph{Moi}.
        Ils n’aimaient tellement pas Notre façon de parler qu’ils Nous ont obligé à parler comme cela.
        \emph{J’ai} fait ci, \emph{J’ai} fait cela.  Comme si Nous étions une personne.

        Remarque, maintenant qu’ils le disent, Nous ne sentons plus vraiment Nos pouvoirs, dans ce petit corps d’homme.
        Nous pouvons encore caresser les oiseaux par le vent, mais Nous ne pouvons plus arrêter le vent que Nous avons créé vers l’île.
        Nous ne pouvons plus modifier les plages et les arbres de l’île.
        Qu’elle étrange impression les hommes doivent ils subir tous les jours.
        Nos oiseaux doivent se sentir libres en comparaison.  Ils ne sont pas tenus de rester cloués au sol, eux.
        Nous voulons voler~!  Nous voulons explorer l’Océan à bord d’une machine volante.
        Nous sentons que si Notre enveloppe humaine meurt, Nous Nous élèverons\ldots et tout reviendras comme avant.
        Mais Nous aurons perdu tant d’énergie dans le processus que Nous ne pourrons plus revenir chez les hommes avant plusieurs générations.

        Nous sommes curieux et Nous voulons voir ce qui va se passer ici, et maintenant.  Profitons en, tant que Nous sommes humain.
        Les hommes se sont séparés en deux camps.  Nous avons été entraîné dans un d’eux et Nous avons obéi.
        Nous ne savions rien faire, Nous ne comprenions pas tous leurs mots techniques.
        Mais il n’ont fait que Nous appeler «~l’ahuri~» et ils ont continué leur démarche folle.

        Contrairement aux autres humains, ces derniers étaient beaucoup plus imposants.
        Ils détruisaient tout sur leur passage pour établir un campement qui n’avait rien à voir en comparaison des cabanes de Nos humains.
        Mais ils étaient beaucoup plus proche du sol que les cabanes.  Nos humains pouvaient au moins se vanter de cela~:  plus proche du sol, moins proche du vol~!
        Ils avaient des armes terribles qui pouvaient toucher à plusieurs centaines de mètres de distance.
        Certains ont essayé de toucher Nos oiseaux, mais Nous les avons arrêtés.
        Tous les soirs, Nous Nous mettions à l’écart pour parler aux oiseaux, ils Nous reconnaissaient.  Peut-être que Nos humain Nous reconnaîtront aussi ?

        Nous n’avions pas vu Nos humains depuis longtemps.  Ils doivent se cacher en hauteur, prudents.
        Un jour, les fugitifs, comme ils s’appelaient eux-mêmes, les ont trouvés.  C’était étrange.
        Chacun des fugitifs les traitait très différemment.
        Il y avait ce «~capitaine~», qui les saluait avec respect, mais qui n’hésitait pas à les menacer dès que l’on d’eux ne se montrait pas serviable.
        Il y avait ce «~classé~», qui les traitait avec un manque de respect incroyable.
        Et d’autres, comme \textsc{José}, en avait de toutes évidences peur.
        De Notre côté, ils semblent ne pas encore Nous avoir reconnu.

        Ce soir, il sera temps de voir ce que Nous pourrons faire pour comprendre ce que ces fugitifs peuvent Nous apporter pour construire une machine volante.
        Mais il s’agit d’ouvrir l’œil et de tâcher de protéger Nos humains et Nos oiseaux contre ces fugitifs dans le même temps~:  cela serait une tragédie que de tout perdre maintenant.
        Voyons ce que Nous pouvons faire.
}

\pageForPlayer{2}{Mayo Manookian, le protecteur de la Terre}
{
    Où est \textsc{Arman}\footnote{Prononcer «~Harmane~».}~?  Où est \textsc{Arman Vezo}~?

    Je m’efforcais de reprendre mes esprits…  Je levais mes yeux aux ciel.
    Ah, c’était ça, j’étais tombé du haut de l’arbre.

    Il arrive.  Super~!
    Lorsque je suis perdu, j’appelle toujours \textsc{Arman} à l’aide.  Il a toujours de bonnes idées et même lorsque je ne comprends pas ce qu’il dit, je tente de faire ce qu’il demande.
    Il est comme ça, \textsc{Arman}, mais je l’aime bien, moi.

    Nous sommes dans la forêt… nous allons bientôt atteindre l’autre bout de l’île~!
    Tiens, \textsc{Arman} s’arrête, on a dû atteindre un endroit important.
    Ou alors il a vu un oiseau, il aime bien les oiseaux.
    Je m’approche.  Oh~?  Une odeur de fumée.

    Serait-ce un autre camp de navigateurs, comme ceux à qui nous avons eu à faire il y a quelques semaines, pas très loin de là.
    Est-ce qu’ils se sont déplacés~?
    C’est fou comment ils prennent possession des terres~!  Moi, une petite cabane me suffit, mais eux, non~!  Il leur faut des bâtiments super compliqué, des constructions mirobolantes…
    À quoi cela peut-il servir~?

    Je suis certain qu’une petite promenade dans les arbres leur ferait du bien… enfin bon, ce n’est pas ce qu’\textsc{Arman} pensera, j’imagine.
    Pourtant, ça aère l’esprit… et puis, un accident est si vite arrivé~!
    C’est vrai, ça~:  un de moins… ça en fera un de moins~!

    Ils arrivent et pouf~!  La plage est à eux, on ne peut plus y marcher sans un laissez-passer, les arbres sont à eux, et ils se permettent de les couper sans raison.
    Sans les arbres, pas d’oiseaux~!  Et les oiseaux, c’est sacré~!
    Les arbres ne leur appartiennent pas~:  il appartiennent à l’Île~!

    Ils ne font preuve d’aucun respect.  Je n’aime pas ça.  Je vais changer cela.
    Ces hommes apprendront le respect de la terre~!  De gré ou de force, ils admireront la nature~!  Ils respecteront l’Île… ainsi que les oiseaux~!

    Et puis, n’oublions pas ce que l’on dit~:  en protégeant la nature, l’immortalité~!  Pouf~!  Comme ça~!
    Mais seuls les meilleurs membre de notre grande famille pourront en profiter.  Du coup, pas touche à la nature ~!

    Ah oui, il faut aussi que je ramène des remèdes pour ma sœur, \textsc{Etha}, qui est malade.
    Le guérisseur ne pouvait rien faire pour elle, mais il pense que les navigateurs pourront nous aider~:  cette maladie n’est ici que depuis qu’ils sont là… ils doivent avoir des remèdes.
    Il faudrait que je le leur demande.

    Mais attention~:  peut-être ma sœur est-elle malade à cause d’eux~!  Si c’est le cas, ils me le payeront~!  Je ne les laisserais pas faire, saccageurs de terre~!

    \paragraph{Analphabète}{
        Bien que parlant inexplicablement la langue des fugitifs, ce n’est pas sa tasse de thé~:  tu ne pourras utiliser aucun mot complexe (mais je suis très souple sur cette notion), à moins d’être en compagnie d’\textsc{Arman Vezo}, ton acolyte qui se débrouille bien mieux que toi dans cette langue de nulle part.
    }
}

\pageForPlayer{3}{Arman Vezo, le prêtre aux oiseaux}
{
	Tiens donc, \textsc{Mayo} se réveille.
	Une seconde sans que je m’en occupe et vlan, le voici à terre.
	Il faut vraiment que je le surveille.
	C’est dommage~:  il est très serviable, il obéit sans broncher à tout ce qu’on lui demande… mais il agit comme un enfant.
	Enfin bon, c’est un excellent ami.

	N’oublions pas notre mission.
	\textsc{Rodrigues} lui même m’a confié une mission importante.
	La récompense~:  l’immortalité.
	Ça ne sera pas la première fois que \textsc{Rodrigues} nous a apporté quelque chose, mais ça, ça reste du jamais vu.

	Pourtant, Il semble rechignant à nous donner plus d’informations.
	J’imagine que cela dérègle sensiblement l’ordre des choses… mais tout de même.
	J’ai parfois l’impression que nous n’arrivons pas à faire ce qu’Il voudrait que l’on fasse, et qu’il soit obligé de nous appâter par ce genre de promesses…
	Il s’agirait de ne pas le dire à hautes voix… ou en tout cas pas à mes acolytes~:  peut-être ses étrangers venus de nulle part ne sont que des agents de \textsc{Rodrigues}~?
	Comment savoir~?  \textsc{Rodrigues} est maintenant silencieux depuis plusieurs mois~!
	Pas un message, pas un appel~!
	D’habitude Il m’appelle régulièrement pour nous indiquer des indices, des choses à aller chercher, des visions diverses… mais là, plus rien.
	Peut-être a-t-Il choisi un autre messager~?
	Peut-être l’ai je déçu~?
	Dans tous les cas, il va s’agir de rattraper cela.

	Juste avant de partir, Il m’a donné le pouvoir de comprendre la langue des nouveaux-venus.
	Je me suis dit que c’était important~:  je l’ai enseigné à quelques amis, dont \textsc{Mayo}.
	Mais bon, en quelques mois avec une langue aussi… complexe~!… il ne faut pas s’attendre à des miracles.

	Mais bon, je m’égard de ma mission.
	Ma mission, la mission qui nous permettra d’atteindre la vie éternelle, consiste à fabriquer un oiseau.
	Oui, fabriquer un oiseau, nous même.
	Faire en sorte que l’on puisse voler.
	Le colorer pour imiter les irisations des magnifiques oiseaux qu’il a créé.
	Et l’utiliser~!

	S’il m’a donné cette langue, c’est pour que je puisse l’utiliser.
	Ces nouveaux hommes ont probablement des choses à nous apprendre et des choses à nous échanger.
	Comment est-il seulement possible de pouvoir supporter son propre poids~?  Les oiseaux le font, mais ils sont légers, eux~!
	Il nous faut de nouvelles connaissances.  Il nous faut tout ce qu’ils savent faire.

	En arrivant à proximité du campement de ces hommes, je reste figé devant leur engin flottant.
	Il est magnifique, immense, imposant… et surtout… il a d’immenses draps blancs.  À quoi cela peut-il bien servir~?
	Et comment quelque chose d’aussi immense peut-il flotter~?
	Pour sûr, ils connaissent des choses que nous ne savons pas.
	Et je pense que ces draps blancs majestueux n’y sont pas pour rien… contre quoi voudront-il bien nous les échanger~?
	Si j’arrive à ramener cela dans notre village, pour sûr, ça sera une étape non négligeable dans la construction de notre engin flottant dans les airs~!
	S’ils ne veulent pas s’en séparer, il me faudra les voler.  \emph{À tous prix~!}
}

\pageForPlayer{4}{Lorenz Orazio, le sédentaire}
{
	Et le voilà qu’il repars~!
	Ce \textsc{Thomas Tew} est d’un énervant, ce n’est pas possible~!
	À peine arrivé sur cette île, nous devons repartir.
	Toujours repartir, repartir, repartir~!

	Il manque de pragmatisme, vraiment.
	J’admets que le rêve de liberté est magnifique, et que libérer tous les esclaves est une bonne chose, mais là…

	Nous sommes arrivés sur cette île par hasard…  Un coup de vent incroyable s’est levé et a emporté les navires de \textsc{Tew} et de \textsc{Misson}.
	Étonnant de constater une levée si rapide de grands vents en plein océan… surtout ici, si loin de l’\textsc{Afrique}, sur la course du \textsc{Gulf Stream} des \textsc{Indes}, qui apporte ses courants chauds vers le Sud.

	Les deux bâtiments, celui de \textsc{Misson} et celui de \textsc{Tew} se sont retrouvé entraînés contre leurs grés, malgré le combat acharné des marins, droit dans cette direction où… Terre~!  Terre~!  Criait \textsc{Rodrigues}, l’idiot du bateau regardant au loin plutôt que d’aider le reste des marins.  Une île sortait des nuages orageux.
	Malheureusement les deux bateaux se sont retrouvés séparés dans la tempête et chacun ont échoués à plusieurs lieux l’un de l’autre.

	Une fois sur l’île, les hommes ont débarqués sur la baie pour y construire un camp de fortune, en attendant un vent plus clément.
	Malheureusement depuis ce temps là, le vent n’a jamais changé de direction.  Cela va faire des mois et des mois que l’on a pas pu reprendre large ici~!  Bientôt une année…  Ce n’est certes plus une tempête, mais impossible d’utiliser le navire.  Les voyages en barques d’un campement à l’autre sont encore possibles, mais très difficiles, ce qui fait que les contacts entre les deux campements sont très rares.

	Une fois sur l’île, les hommes ont eu deux types de réactions~:
	\begin{itemize}
		\item Certains venaient d’une vie de marin non voulue, enrôlés pour une guerre ou par manque d’argent et ne se rendant pas compte combien la terre est peu vue sur l’océan.  D’autres venaient tout juste d’être libérés de l’esclavage par le capitaine \textsc{Misson} et redécouvrait voire découvrait la signification de la liberté.
			Ces hommes se sont empressés d’utiliser l’échappatoire à la marine qui leur était proposé pour prendre des «~vacances mérité~» dans ce camp de fortune.
		\item D’autres voulaient reprendre le rythme militaire.  Parfois par habitude, mais surtout car les plus cultivés savaient se qui s’était passé leur de la colonisation de l’Amérique~:  les premières années, le cannibalisme n’était pas rare, dû au manque de nourriture, lui même dû au fait qu’aucun champ de céréales n’était prêt les premiers mois de la colonisation.
			Ces hommes ont tout fait pour réorganiser une société hiérarchisé dans le camp… est c’est probablement grâce à eux que celui-ci existe encore.
	\end{itemize}
	Dans tous les cas, aucune exploration sérieuse de l’île n’a pu être exécutée… et tous les hommes envoyés chercher un chemin terrestre entre les deux campements sont soit rentrés bredouille, soit pas rentrés du tout.

	Une société de fortune a donc été mise en place par ces fugitifs, autoproclamés «~hommes libres~».
	Tu parles d’une société~!  La moitié veux déjà repartir~!
	Non mais il vous arrive de faire preuve de pragmatisme, parfois~?!?
	Si vous revenez sur le continent vous serez accusés de piraterie — à juste titre, d’ailleurs — et vous n’en reviendrez pas vivants.

	Il est temps d’abandonner ses discours à la noix et de s’installer ici, tentant de vivre malgré la difficulté.
	Nous fonderont une nouvelle société isolée et tranquille.

	Lorsque \textsc{Tew} a demandé des volontaires pour aller prévenir \textsc{Misson} qu’il était temps de tout préparer pour repartir dès que le vent se lèvera — s’il se lève —, je me suis tout de suite porté volontaire pour faire le trajet.
	C’est risqué, mais je préfère interférer maintenant avec ces idées à la noix et mentir à ce \textsc{Misson}~:  nous resterons ici jusqu’à nouvel ordre~!
	Pourquoi ne pas aller vivre avec les autochtones~?  Il faut essayer, non~?

	Aïe, par contre, je vais devoir convaincre mon acolyte \textsc{Klaas Goch} de changer le message.
	Ça par contre, ça ne vas pas forcément être de la tarte…
}

\pageForPlayer{5}{Klaas Goch, l’explorateur}
{
	\paragraph{Prononciation}{ Tu es d’origine néerlandaise~:  si tu ne sais pas parler le néerlandais, merci de demander au MJ (moi — \textsc{Martin} — donc) comment se prononce mon nom. }

		Lorsque mon capitaine m’a demandé d’aller botter un peu les fesses de ce capitaine \textsc{Misson}, j’étais tout excité~!
		Je vais pouvoir faire d’une pierre deux coups.

		Cela fait trop longtemps que nous sommes restés au même endroit… il est temps d’explorer~!
		Une île inconnu dont les vents semblent surnaturels… cela rime forcément avec trésors~!
		Je veux explorer cette île, je veux découvrir ses secrets les plus fous.
		Je veux trouver ce que les indigènes ont caché de plus précieux dans la forêt.
		Et je suis vraiment prêt à tout pour cela~:  jamais, \emph{jamais}, personne n’a exploré cette île… s’il y a quelque chose, on ne partagera pas.  Et il y a forcément quelque chose.  Il ne me reste qu’à le trouver.

		Aller traverser le pan d’eau entre les deux campements, malgré le danger, sera l’occasion d’explorer plus avant.
		Ça sera aussi l’occasion d’en découvrir un peu plus sur ce que cache ces sauvages~:  le capitaine \textsc{Tew} a bien insisté pour que je n’approche pas des indigènes sur son campement… mais rien n’a jamais été dit quant à celui de \textsc{Misson}~!
		On ne me laisse pas à l’écart si facilement que cela, moi~!

		Et puis aller botter les fesses du capitaine \textsc{Misson}~!  La grande classe~!
		J’avais toujours envie de le lui dire en face… et il va être temps~!
		Lui, qui rêvasse toujours à sa société libre et tout ça… mais on oublie le plus important~:  nous sommes des pirates, bougre de bougre~!
		Et nous allons le lui faire comprendre~!
		Qu’ils reprennent flots tout de suite ; nous allons longer la côte pour attaquer et piller les indigènes sur leur propre terrain~!

		Hé hé hé, le capitaine \textsc{Tew} a bien fait de m’envoyer~:  on arrive au campement ; dans une heure, c’est réglé.
}

\players[4][5]{
	\paragraph{À noter}{ \textsc{Thomas Tew} a déjà pu négocier de bonnes relations avec les indigènes (qui sont certes sur des relations du type «~chacun de son côté~» avec quelques communications ou échanges additionnels)~:  le capitaine \textsc{Misson} est donc clairement en retard de ce côté là. }
}

\pageForPlayer{6}{Angelo Caraccioli, le missionnaire}
{
		Ah~!  Que ma vie a changé depuis qu’ \textsc{Olivier Misson} a décidé de prendre son indépendance totale.
		Nous sommes maintenant libres, complètement libre~!
		J’étais grenadier du roi sur la \textsc{Victoire}… un soldat tuant hommes et biens.
		Me voici un homme libre capable de prouesses et de bonté dans une ville utopique.

		J’avais fait confiance à \textsc{Olivier} pour ses idées d’avant garde, mais là, il dépasse mes rêves les plus fous~!
		Se déclarer indépendant, se lancer dans une lutte sans fin contre l’esclavage~!
		Pour sûr Dieu est avec nous~!  Quoi de plus divin comme quête que celle de la liberté pour tous~!

		Cependant, j’ai parfois l’impression que l’on a tendance à tomber en décadence ici et à un petit peu oublier cette divinité à qui l’on doit tout.
		C’est pourtant d’une importance majeure~!  Sans Dieu, c’est le chaos, la destruction.
		Les autres religions ne font que se battre, Dieu leur apporte l’unité~!
		Dieu nous apporte un conseil, nous montre le vrai chemin, celui que nous suivons maintenant.
		Il ne faudrait pas l’oublier~!

		Il est temps que je fasse quelque chose pour changer cela.
		Mais il va s’agir de le faire calmement~:  la décadence prévue par la Bible est en train de s’abattre sur nous.
		Et Dieu l’a vu~!  Ce vent qui sans arrêt souffle vers les côtes, c’est un message~!
		Il souhaite que l’on se ressaisisse~!  Priiez Dieu, mes amis~!  C’est l’heure~!

		Aïe, aïe, aïe, ils ne m’écoutent pas…  Bon, il va décidement falloir jouer plus finement et plus calmement.
		Fois de \textsc{Caraccioli}, ces hommes prierons, et rapidement~!

		Et que voici donc… des indigènes.  Mon rôle de missionnaire va pouvoir se mettre au goût du jour~!
		Je vais pouvoir leur montrer la vérité, la lumière.
		Tout l’air de ce genre de missions et d’y aller progressivement.
		Mais ils vont y croire, vous allez voir~!

		Maintenant que j’y pense… un message, ce vent~?
		Il est peut-être présent depuis trop longtemps, d’habitude Dieu n’est peut-être pas si insistant…
		Peut-être est-ce plutôt une menace… ou un message du Diable lui-même~!
		La plupart des légendes mentionnant le Diable le font prendre possession du corps des hommes~:  j’ai peur de ce qui va arriver (ou de ce qui est déjà arrivé)… il prendra très probablement possession du corps du marin le plus décadent — ou des indigènes non encore convertis à l’esprit plus malléable~!
		Vite~!  Il faut faire quelque chose~!
		\textsc{Olivier Misson}~!  Je dois vous parler de toutes urgence~!
}

\pageForPlayer{7}{Olivier Misson, le révolutionnaire}
{
		Et bien ce n’était pas facile, mais j’y suis finalement arrivé~!
		\emph{\textsc{Libertalia}} est enfin créée~!
		Nous sommes libres~!

		Je me souviens encore de mon discours d’arrivé ici~:  «~Notre cause est une cause noble, courageuse, juste et limpide~:  c’est la cause de la liberté.  Je vous conseille comme emblème un drapeau blanc portant le mot «~Liberté~», ou si vous la préférez, cette devise~:  «~A Deo a libertate~», «~Pour Dieu et la liberté~».  Ce drapeau sera l’emblème de notre infaillible résolution.  Les hommes qui sauront prêter une oreille attentive aux cris de~: «~Liberté, liberté, liberté~» en seront les citoyens d’honneur.~»
		Puis nous avons tous criés «~Liberté, liberté, liberté~!~»
		S’était magnifique~!

		Et nous avons crié cet hymne à chaque fois que l’on pillait des navires pour reprendre ce qui a été volé aux esclaves, ainsi que pour affranchir tous les esclaves que l’on capturait.
		Et il y en avait~!

		Pourtant, depuis depuis que le second campement, celui de \textsc{Thomas Tew} semble avoir moins de problème, les tempéraments se sont affaiblis de mon côté.
		Le fait qu’il y ait ainsi deux \textsc{Libertalia} différentes n’est certes dû qu’à des raisons logistiques, cela affaibli très nettement les deux parties.
		Le pire reste cependant le fait que l’on ne puisse aller de l’un et de l’autre que par navire, si l’on souhaite éviter les zones inexplorées de \textsc{Rodrigues}, l’île où nous nous sommes arrêté.

		Un jour sera temps d’explorer… mais nous verrons ça après.
		Pour l’instant, il y a plus important~:  des membres du campement de \textsc{Thomas Tew} sont arrivés aujourd’hui pour apporter des nouvelles… mais c’est précisement à ce moment qu’arrivent deux indigènes~!
		Il va s’agir de jouer serrer pour que tout fonctionne~!
}

\pageForPlayer{8}{José Felipe, le boulet}
{
	Encore une aventure~!
	Ce n’est pas possible… je porte la poisse.
	Il me suffit d’être dans un navire pour qu’il se fasse chasser.
	Et c’est ce qui nous est arrivé…
	Et où a tiré le \textsc{Winchelsea}~?  En plein dans la dunette~!
	Tous les officiers étaient à cet endroit…
	\textsc{Misson} n’était qu’aspirant et a ainsi accédé au rang de capitaine… et moi de second.

	C’est fou~!  Moi qui n’ai jamais rien su faire de ma vie~!
	Moi qui ai toujours été maladroit avec mes mains, mes gestes et mes dires~!
	Si l’on m’avait placé à ce poste de second aspirant, c’est précisément parce que tout le monde sait que ce poste ne sert strictement à rien dans un bâtiment~!
	Mais bon, si on me mettait à laver le pont, quelqu’un glissait et tombait à la mer.
	Si on me demandait de m’occuper d’un canon, il finissait par tirer et fracasser un mur.
	Si on me mettait en assistance du docteur lors de la bataille, j’ouvrais la porte en claquant alors qu’il opérait un marin… bousculant le médecin en passant.  Un marin a perdu son bras à cause de moi dans l’événement~:  il a voulu me tuer en attrapant un des couteaux du médecin et me viser.
	Je me suis caché derrière le médecin qui a reçu le couteau en plein cœur.
	On m’a mis dans ce poste complètement inutile parce que je porte la poisse~!
	Quelque soit ce que je fasse…

	Et qu’est ce qui s’est passé par la suite~?
	Et vlan, nous voilà coincé dans une tempête~!
	Je vous dit~:  je deviens second par hasard et pouf~!  Une tempête~!

	Je n’aurais jamais dû partir en mer pour annuler mes dettes.
	Je me suis retrouvé en plein milieu d’une île déserte où les gens sont devenus complètement dingues.
	Le climat complètement fou leur en a d’ailleurs complètement fait oublié ma poisse…
	Tous le monde parle de liberté, de quitter les Empires, de fonder un peuple libre.

	Mais vous vous rendez-compte de ce que vous dites~?!?
	Et maintenant, que faites vous~?  Vous parlez avec des indigènes.  Des indigènes~!
	Ils sont dangereux, ils cachent quelque chose, j’en suis certain~!  Avec ma chance, c’est même plus que certain.

	Mais… mais… s’ils cachent quelque chose, c’est peut-être précieux~?
	Si j’ai une poisse pareil, c’est bien pour quelque chose, non~?
	Un jour, il est temps de l’utiliser~!

	Si ces indigènes cachent quelque chose, c’est probablement une de ces choses que l’on raconte dans les histoires — et j’en connais beaucoup des histoires~!
	On parle par exemple d’une pierre donnant l’immortalité se trouvant sur une île présente sur aucune carte proche de \textsc{Madagascar}… cela semble correspondre énormément et j’aimerait bien vérifier cela auprès des indigènes… mais discrètement~:  ils sont dangereux~!
	Pour une fois que ma position de second me sert à quelque chose… je pourrait même trouver des gens pour me suivre dans l’aventure~!
}

\pageForPlayer{9}{Horace Ravin, l’amasseur de trésors}
{
	Depuis que nous sommes ici, les ressources commencent à manquer.
	Nous n’avons toujours pas de champs de céréales ni d’animaux d’élevage, et la cueillette et la chasse, ça va bien quelques semaines, mais pas plusieurs mois~!
	Avec ce que nous avons là, nous pourrons à peine tenir deux mois supplémentaires.
	Et les gens ne s’en rendent pas compte~!
	Ils sont fous~!  Le simple fait d’être sur une plage n’empêche pas les cerveaux de compte, que diable~!
	Il semblerait que si… à moins que ça ne soit l’abus d’alcool, principal utilisation des fruits de la cueillette~?

	J’ai peur que les ressources que nous avons ne soient pas suffisantes pour tenir et j’aimerais pointer du doigt des «~coupables~».
	Hors de question de le faire publiquement — je n’aimerais pas que le manque de ressource se sache — il me suffit de trouver des hommes un peu plus faibles que les autres et les piéger.
	Leur faire croire qu’ils ont fait quelque chose de grave… puis leur faire du chantage.
	Si je me débrouille bien, mes cibles ne voudront \emph{vraiment} pas que je répande la rumeur comme quoi il n’y a plus de nourriture.
	Qu’ils aillent chercher de nouvelles ressources… et qu’ils ne dépendent plus du camp~!  Une personne de moins, c’est une personne de moins à nourir.

	Tiens par exemple ce \textsc{Rodrigues}…  Ah~!  Ce \textsc{Rodrigues}~!
	L’ahuri, comme je préfère l’appeler.
	Il ferait tout ce que je lui demande~:  il me suffit de lui faire suffisamment peur~!
	Mais il cache des choses, j’en suis certain~!  Et bien il en aura encore plus à cacher s’il ne me dit pas tout~:  je vais lui faire croire des choses qu’il n’aura jamais imaginé avoir fait, ah, ah, ah~!

	Et voilà des hommes du camp d’à côté qui arrivent~! Ah, non~!  Je refuse toute nouvelle bouche à nourrir~!  Il est temps d’instaurer une bourse au mérite sur les rations du camp.

	En parallèle, il serait temps d’utiliser ma position de gestionnaire des ressources pour augmenter son pouvoir… quitte à réaliser mon rêve d’entant et devenir capitaine, quelqu’en soient les conséquences.
	En période de pénurie, ceux qui ont accès aux ressources sont rois~:  ma position n’est clairement pas à son potentiel~!

	Je n’ai fait qu’une seule attaque de navire\ldots mais c’était magique~!
	Vivre au dépend total des autres, avoir une vie d’aventures et d’amusements, sans labeur.
	J’adorerais pouvoir relancer une attaque un de ces jours.

	Tiens ben… pourquoi pas contre le navire de \textsc{Tew}, l’\textsc{Amity}~?  Il est juste à côté et il ne nous attend pas.
	Je ne sais pas comment je pourrais convaincre \textsc{Misson} de faire cela, mais ça vaudrait le coup… pourquoi ne pas utiliser son second~?
}

\pageForPlayer{10}{Philippe Cosnac, l’infiltré}
{
		Je suis en fait un infiltré français travaillant pour la couronne~:  à la suite d’une expérience pour limiter la piraterie, chaque navire s’est vu équipé d’un «~mouchard~» — comme moi~!
		Ces hommes seront immédiatement promus à un poste de rentier à vie, suffisamment pour pouvoir s’acheter la villa de leur rêve dans l’endroit de leur choix et y vivre tranquillement jusqu’à la fin de leur vie, s’il arrivent à dénoncer (en donnant les coordonnées des positions des navires à n’importe quel membre de la couronne) les capitaines qui ont sombré dans la piraterie.

		Pour l’instant, j’essaye de survivre, bien entendu… je porte un tatouage qui identifie ce type d’infiltrés et je ne doit pas le montrer, sous aucun prétexte.
		Sauf bien sûr si j’arrive à me faire des alliés.
		Mais bon, ce n’est pas gagné~:  tout le monde ici crie «~liberté~» pour la moindre raison.

		Cependant, grâce à ce vent continu, les hommes commencent à se lasser, à rentrer en décadence.
		Il est temps pour moi de les forcer à reprendre le large malgré le vent.
		Il est temps de revenir chez nous… et de profiter de ma rente~!

		Et pour les convaincre de partir… il me suffit de faire accélérer le processus de décadence.
		Il me faut les dégoûter suffisamment de cette île pour qu’ils partent.
		Il me faut saboter toutes les actions qu’ils entreprennent.
		Mais attention à ne pas se faire prendre~!

		Mon but est que suffisamment de marins soit motivés pour prendre le large, avec moi en capitaine de préférence, mais ce n’est même pas obligé~:  ce qui compte, c’est que je puisse retourner sur le continent… et alors avertir les autorités~!
		J’ai une rente à gagner, moi… contrairement à tous ceux qui me suivront, qui auront la tête tranchée — mais ça, inutile de le leur dire, bien au contraire~!
}

\pageForPlayer{11}{Étienne Pallière, le docteur}
{
        Aïe, aïe, aïe\ldots  Pas joli, joli, cette jambe.
        Des boutons comme cela, ça n’est pas très naturel~; mais qu’est ce que ça peut être~?
        Un début de lancerquèm\footnote{«~cancer~» en louchébem.}~?  Est-ce qu’il faut amputer~?

        Amputer, amputer\ldots je ne sais faire que cela.
        Je ne suis pas docteur, moi~!  J’étais charcutier, à l’époque.
        Je faisais même partie de ceux qui insistaient pour garder le louchébem et pour l’enseigner aux nouveaux.
        Ah\ldots c’était le bon temps~!

        Mais depuis, il y a eu cette catastrophe.
        Un jour, une\ldots porte~; je pense que c’est le meilleur mot, oui, une porte venue d’ailleurs~; est apparue.
        Une lumière bleue très forte, et ce trou, cette porte, dans l’espace temps\ldots enfin tout ce qu’ils en disent, les scientifiques qui n’y connaissent rien.
        Ils n’y connaissent vraiment rien, ça non.
        Jamais je n’avais entendu parler de quelque chose comme ça.
        La porte a fait un appel d’air, je me suis retrouvé happé.

        De l’autre côté, du sable, à perte de vue.
        La porte avait disparue et je me suis retrouvé de l’autre côté.
        Passé la phase d’étonnement, je me suis mis à explorer alentours.
        C’était l’Égypte\ldots mais pas n’importe laquelle~:  l’Égypte \emph{ancienne}.
        La porte m’avait fait faire un bond de plusieurs milliers d’années dans le temps~!

        J’ai dû survivre.  Pas si facile, mais je m’en suis sorti d’affaire.
		L’avantage d’un pays avec autant d’esclaves est que l’on ne pose pas de question lorsque quelqu’un ne parle pas la langue…  Parce qu’avec mon français… lomprenquès dans le lap~!
		C’était très douloureux, mais je suis parvenu à m’en sortir… non sans quelques séquelles.

        Je revenais régulièrement à l’endroit où la porte m’avait projeté.
        Quasiment un an avant après (10 mois et 3 jours pour être précis), la porte est revenue.
        Elle semblait calme, contrairement à la première fois, comme si elle attendait quelque chose.

        Je ne fis ni une ni deux~:  je voulais rentrer dans un monde civilisé, où les malades n’étaient pas simplement mis à l’écart en attendant leur mort.
        Je voulais revoir mes années 1980~!
        Je me précipitais dans le gouffre temporel\ldots

        \ldots pour me retrouver en plein champ de bataille, au Moyen-âge~!
		C’était l’horeur~!
		Les morts tombaient les uns après les autres…
		N’appartenant clairement à aucun camp — qui se trimballe en sahourie alors qu’il se les gèle dehors à part un fou (ou un voyageur temporel perdu loin de chez lui…)~? — personne n’avait d’agressivité particulière à mon égard.  On préfère toujours taper sur des gens armés qui nous agresse que sur le fou du coin.

		J’ai pris mes jambes à mon cou.  J’ai pu voir un chevalier ensanglanté tomber à terre de son cheval.
		Je l’ai aidé à se relever.  Il était totalement abasourdie de voir que finalement, il allait s’en sortir~:  il n’était pas blessé, mais le poids de son armure était bien trop élevé pour qu’il puisse faire quoi que se soit sans l’enlever.
		Il a alors brandi son épée pour m’éviter un coup fatale venant de derrière mon épaule~!
		Je lui dois la vie.
		J’ai alors pris une dague sur le corps d’un cadavre et nous avons combattu ensemble.
		Il frappait les ennemis et je les finissaient.
		Mine de rien, mous étions efficaces.

		Après des heures de combat intense, nous avons finalement survécu.
		Le chevalier m’a offert sa protection et le loger.
		Nous sommes devenu amis.

		Cependant, j’étais toujours choqué de part la traversée de cette porte…
		La survie ici était plus facile… mais l’hygiène n’y était pas.
		J’ai presque envie de dire que c’était pire qu’en Égypte~:  là-bas au moins, la sécheresse évitait aux eaux stagnantes de mal tourner…

		Je vivais plutôt reclus, j’évitais de voir du monde dès que je le pouvais.
		Mon ami de fortune m’avait offert suffisamment d’or pour pouvoir me payer un lit et un couvert pendant plusieurs mois (du moins si l’on acceptais de se mettre au niveau de qualité demandé par le peuple…).
		J’avais pu ainsi me faire passer comme docteur de la dernière chance.
		J’amputais les membres lorsqu’ils commençaient à devenir noirs…
		Pas un beau métier… mais je ne vacillais pas à la vue du sang et savais amputer correctement — c’est à dire qu’après l’amputation je ne passais pas un quart d’heure à jouer avec les os encore connectés au système nerveux pour panser~:  la plupart du temps, les cris postopératoires semblaient pire que ceux de l’amputation elle-même~!  Du coup quelqu’un qui sait s’y prendre, c’est plus qu’apprécié, même s’il a un accent de fou.

		Un jour, l’inquisition a eu vent de moi…
		Un étranger, qui se prend pour un médecin (alors que la plupart des docteurs de l’époque étaient des moines) et qui ne fait «~pas~» souffrir ses victimes, c’était forcément l’œuvre du diable.
		Forcément…
		Ces religieux…

		J’étais pris dans une chasse à l’homme — une de leurs battues frénétiques…
		C’est alors que je me souvenais que cela faisait maintenant plus de 10 mois que j’étais ici~!
		Frappé par cet éclair de lucidité, je me suis mis à courir en direction de l’endroit où j’étais apparu.
		Un regard en arrière.  Mes agresseurs se rapprochaient… lorsque je trébuchais et perdis connaissance.

		Je sentais le sable… le soleil me chauffait le visage.
		J’étais passé par la porte~!
		Je craignais les pyramides… du sable, ce n’est pas bon signe.
		J’ai alors ouvert les yeux.
		J’étais sur une île… une île d’«~hommes libres~», comme ils s’appellent eux-mêmes.

		Une bonne chose~:  en $1674$, le français des années $1980$ (si on y ajoute pas trop de louchébem, cependant) commence à ressembler à leur largomuche \emph{lingua franca}.  Je pouvais enfin communiquer~!
		Je me lis d’amitié avec les locaux.
		Après avoir vu l’asservissement de l’Égypte ancienne, et en ayant moi même fait les frais, je ne peux qu’adhérer à leurs idées de libération des esclaves.

		Voilà maintenant $10$ mois que je vis en tant que médecin ici.
		Ma faiblesse de l’utilisation de l’amas de langues locales\footnote{En pratique, il suffira de glisser quelques mots de louchébem en jeu.} a fait que tout le monde m’appelle le «~docteur~», sans plus de dénomination.
		De toutes façons, vu le nombre de médecins ici… ça ne changera pas grand chose.
}

\end{document}

