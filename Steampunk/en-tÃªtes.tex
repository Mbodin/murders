% Créé par Martin Bodin (2011).
% Document sous licence CC BY-NC-SA

\documentclass{article}
%\documentclass{scrartcl}

\usepackage{ifxetex}
\ifxetex
\usepackage{xunicode,fontspec,xltxtra}
\else
\usepackage[utf8x]{inputenc}
\usepackage[T1]{fontenc}
\usepackage{amsmath, amsthm}
\usepackage{amsfonts, amssymb}
\fi

\usepackage[francais]{babel}
\usepackage{lmodern}
\usepackage{stmaryrd}
\usepackage{graphicx}
\usepackage[nottoc, notlof, notlot]{tocbibind}
\usepackage[dvipsnames]{pstricks}
\usepackage{pst-circ, pst-plot, pstricks-add}
\usepackage{array}
\usepackage{url}
\usepackage{verse}
\usepackage[colorlinks,linkcolor=black]{hyperref}
\usepackage{ifthen}
\usepackage{longtable, rotating}
%\usepackage{fancyhdr}
\usepackage{fancybox, framed}
\usepackage{textcomp}
\usepackage{marvosym}
%\usepackage{bbding}
%\usepackage{a4wide}
\usepackage{geometry}
%\usepackage{soul}
\usepackage{lettrine}
%\usepackage{yfonts}
\usepackage{oldgerm}
\usepackage{enumerate}
\usepackage{tikz}
\usepackage{dictsym}
\usepackage{pifont}

\ifxetex
\newfontfamily\timesfont[Ligatures=TeX]{Times New Roman}
\setmainfont[Mapping=tex-text, Ligatures={Contextual, Common, Historical, Rare, Discretionary}, Numbers={OldStyle}]{Linux Libertine O}
\fi

%\newcommand{\enluminure}[2]{\lettrine[lines=3]{\small \initfamily #1}{#2}}

\usetikzlibrary{trees}
\usetikzlibrary{arrows,shapes,automata,petri}
\usetikzlibrary{fit}
\usetikzlibrary{calc,decorations.pathmorphing,patterns}


\geometry{
	includeheadfoot,
	margin = 2.54cm,
	top = 1.5cm,
	bottom = 1.5cm
}

\newcommand{\ds}{\displaystyle}

\renewcommand{\ge}{\geqslant}
\renewcommand{\le}{\leqslant}
\renewcommand{\preceq}{\preccurlyeq}
\renewcommand{\succeq}{\succcurlyeq}

\newcommand{\Numero}{\No}
\newcommand{\numero}{\no}

\newcommand{\fixme}{\textbf{FIXME}}

\makeatletter

\newcommand{\defineNewPlayer}[2]{
	\@namedef{couleur#1}{#2}
}

\newcommand{\getPlayerColor}[1]{%
	\@nameuse{couleur#1}%
}

\makeatother

% Des commandes pratiques pour générer le document.
\newcommand{\player}[2]{%
	\ifthenelse{\equal{\forplayer}{y}}{%
		\ifthenelse{\equal{\theplayer}{#1}}%
		{#2}{}%
	}{\begin{barv}[\getPlayerColor{#1}]{2pt}{10pt}#2\end{barv}}%
}
\newcommand{\mj}[1]{%
	\ifthenelse{\equal{\forplayer}{n}}{#1}{}%
}
% Ici suit une commande plus complexe, car plus générale.
\makeatletter

\newcommand{\@beginColor}[3][black]{%
	\ifthenelse{\equal{\forplayer}{n}}{%
		\begin{barv}[#1]{#2}{#3}%
	}{}%
}

\newcommand{\@endColor}{%
	\ifthenelse{\equal{\forplayer}{n}}{%
		\end{barv}%
	}{}%
}


\newcommand{\ignore}[1]{}
\newcommand{\@ident}[3]{%
%	\ifthenelse{\equal{\manyColored}{y}}{#1}{%
%		\marginpar{%
%			#1%
%			\vspace{2cm}%
%			#2%
%		}%
%	}%
	#1%
	\ifthenelse{\equal{\forplayer}{n}}{\@beginColor{0pt}{10pt}}{}%
	#3%
	\ifthenelse{\equal{\forplayer}{n}}{\@endColor}{}%
	#2%
%	\ifthenelse{\equal{\manyColored}{y}}{#2}{%
%		\marginpar{%
%			#1%
%			\vspace{5pt}%
%			#2%
%		}%
%	}%
}

\def\@ouverture#1#2{%
\ifthenelse{\equal{\forplayer}{y}}{}{%
\ifthenelse{\equal{\manyColored}{y}}{\@beginColor[#1]{1pt}{0pt}}{%
\hspace{-1cm}\hspace{-#2mm}\parbox[c][1pt][t]{0pt}{
\begin{tikzpicture}
	\node (a) {};
	\node (b) [right of = a, node distance = 16cm] {};
	\node (c) [below of = a, node distance = 2cm] {};
	\draw [very thick, color = #1] (a.center) -- (b);
	\draw [very thick, color = #1] (a.center) -- (c);
\end{tikzpicture}
}\vspace{-3.2mm}\par%
}%
}%
}
\def\@fermeture#1#2{%
\ifthenelse{\equal{\forplayer}{y}}{}{%
\ifthenelse{\equal{\manyColored}{y}}{\@endColor}{%
\hspace{-1cm}\hspace{-#2mm}\parbox[c][1pt][b]{0pt}{
\begin{tikzpicture}
	\node (a) {};
	\node (b) [right of = a, node distance = 16cm] {};
	\node (c) [above of = a, node distance = 1cm] {};
	\draw [very thick, color = #1] (a.center) -- (b);
	\draw [very thick, color = #1, dashed] (a.center) -- (c);
\end{tikzpicture}
}\vspace{-3.2mm}\par%
}%
}%
}

\def\players@parse#1#2[#3][#4]{%
% #1 :  Suite de \@ouverture
% #2 :  Suite de \@fermeture
% #3 :  Commande à appeler dans le cas d’une réponse négative (≃ réponse précédente).
% #4 :  Argument (sous forme de numéro de joueur) lu actuellement.
	\ifthenelse{\equal{\theplayer}{#4}}{%
		\players@yes{\@ouverture{\getPlayerColor{#4}}{#4}#1}{#2\@fermeture{\getPlayerColor{#4}}{#4}}%
	}{%
		#3{\@ouverture{\getPlayerColor{#4}}{#4}#1}{#2\@fermeture{\getPlayerColor{#4}}{#4}}%
	}%
}

\def\players@no#1#2{%
	\@ifnextchar[{\players@parse{#1}{#2}[\players@no]}{\ignore}%
}

\def\players@yes#1#2{%
	\@ifnextchar[{\players@parse{#1}{#2}[\players@yes]}{\@ident{#1}{#2}}%
}

\def\players{%
	\ifthenelse{\equal{\forplayer}{y}}{%
		\players@no{}{}%
	}{%
		\players@yes{}{}%
	}%
}

% \players{…} est quasi-équivalent à \mj{…}.
% \players[i]{…} est équivalent à \player{i}{…}
% \players[i][j][k]{…} va créer du contenu uniquement pour les joueurs i, j et k (et les MJ bien sûr).

\makeatother
%\fixme :  Ces commandes posent des problèmes pour toutes les sections, footnote, etc. :S

\newcommand{\colorForMJ}[2]{%
	\ifthenelse{\equal{\forplayer}{y}}{#2}{%
		\textcolor{\getPlayerColor{#1}}{#2}%
	}%
}
\newcommand{\synopsisPerso}[3]{%
\paragraph{}{
\textbf{\fcolorbox{\getPlayerColor{#1}}{white}{#2}}\hspace{10pt}%
{#3}}%
}

\newenvironment{changemargin}[2]{\begin{list}{}{%
\setlength{\topsep}{0pt}%
\setlength{\leftmargin}{0pt}%
\setlength{\rightmargin}{0pt}%
\setlength{\listparindent}{\parindent}%
\setlength{\itemindent}{\parindent}%
\setlength{\parsep}{0pt plus 1pt}%
\addtolength{\leftmargin}{#1}%
\addtolength{\rightmargin}{#2}%
}\item }{\end{list}}
\reversemarginpar
%\pagestyle{fancy}
%\fancyhf{}
%\renewcommand{\headrulewidth}{0pt}
%\lhead{}
%\lfoot{}

\makeatletter
\newenvironment{barv}[3][black]{%
% #2 largeur du trait
% #3 distance entre le trait et le texte
	\def\FrameCommand{{\color{#1}\vrule width #2}
	\hspace{#3}}%
	\MakeFramed {\advance \hsize -\width \FrameRestore }%
}{%
    \endMakeFramed%
}
\makeatother


\definecolor{LightButter}{rgb}{0.98,0.91,0.31}
\definecolor{LightOrange}{rgb}{0.98,0.68,0.24}
\definecolor{LightChocolate}{rgb}{0.91,0.72,0.43}
\definecolor{LightChameleon}{rgb}{0.54,0.88,0.20}
\definecolor{LightSkyBlue}{rgb}{0.45,0.62,0.81}
\definecolor{LightPlum}{rgb}{0.68,0.50,0.66}
\definecolor{LightScarletRed}{rgb}{0.93,0.16,0.16}
\definecolor{Butter}{rgb}{0.93,0.86,0.25}
\definecolor{Orange}{rgb}{0.96,0.47,0.00}
\definecolor{Chocolate}{rgb}{0.75,0.49,0.07}
\definecolor{Chameleon}{rgb}{0.45,0.82,0.09}
\definecolor{SkyBlue}{rgb}{0.20,0.39,0.64}
\definecolor{Plum}{rgb}{0.46,0.31,0.48}
\definecolor{ScarletRed}{rgb}{0.80,0.00,0.00}
\definecolor{DarkButter}{rgb}{0.77,0.62,0.00}
\definecolor{DarkOrange}{rgb}{0.80,0.36,0.00}
\definecolor{DarkChocolate}{rgb}{0.56,0.35,0.01}
\definecolor{DarkChameleon}{rgb}{0.30,0.60,0.02}
\definecolor{DarkSkyBlue}{rgb}{0.12,0.29,0.53}
\definecolor{DarkPlum}{rgb}{0.36,0.21,0.40}
\definecolor{DarkScarletRed}{rgb}{0.64,0.00,0.00}
\definecolor{Aluminium1}{rgb}{0.93,0.93,0.92}
\definecolor{Aluminium2}{rgb}{0.82,0.84,0.81}
\definecolor{Aluminium3}{rgb}{0.73,0.74,0.71}
\definecolor{Aluminium4}{rgb}{0.53,0.54,0.52}
\definecolor{Aluminium5}{rgb}{0.33,0.34,0.32}
\definecolor{Aluminium6}{rgb}{0.18,0.20,0.21}

\pgfdeclarelayer{foreground} 
\pgfdeclarelayer{background} 
\pgfsetlayers{background,main,foreground} 

