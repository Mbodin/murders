\input{../en-têtes.tex}

\title{\textgoth{M}onoplane à vapeur}
\author{\textsc{John Stringfellow}}
\date{}

\begin{document}

\maketitle

\lettrine{\textgoth{U}}{ne amélioration} d’une version précédente où la portée des ailes est suffisante pour acceuillir une machine à vapeur.
La quantité de charbon et d’eau transportable sur une telle machine devrait avoisiner les trentes kilogrammes, ce qui est suffisant pour environ une heure de vol.
Nous recherchons les moyens d’augmenter cette valeur afin que la machine puisse effectuer des déplacements industriels.

\begin{center}
	\includegraphics[width = 18cm]{engin.png}
\end{center}

\begin{enumerate}[A]
	\item Câbles de contrôles de l’assiette ;
	\item Évacuation de la vapeur ;
	\item Ordinatrice minimaliste calculant les paramètres de la machine ;
	\item Contrôle directionnel, actionné à la main ;
	\item Hélices, actionnées par la machine à vapeur.
\end{enumerate}

\[
	F = \frac{\rho V^2 S C}{2}
\]
avec $\rho \approx 0,9877 \mathrm{kg/L}$, $S = 20m^2$, $C = \displaystyle\int_1^\infty\! \frac{S}{x^2 \times 2,3m^2}\mathrm{d}x \approx 8,696$ et $V = 72 \mathrm{mph}$.

\end{document}

